\chapter{Continuous gravitational wave injections}

%%%%%%%%%%%%%%%%
In this section I outline how we inject a \ac{CW} signal into data. This can generally be done in two different ways: simulating a signal in the time domain and injecting into time domain noise or simulating the power spectrum of a signal and injecting the signal into a \ac{PSD}.
%%%%%%%%%%%
\subsection{CW Signal}
%%%%%%%%%%%
This section has been covered in Sec.~\ref{intro:cw:signal}, The CW signal in generated by using a lalsuite package where ....

%%%%%%%%%%%%
\subsection{Time series and complex \ac{FFT} injections}
%%%%%%%%%%%%
Injections into time-series data is relatively simple. Given a set of parameters for the source the signal can be generated in the time-series, this is then just summed with the time-series which it is injected into. Similarly with the \ac{FFT}, the time-series of the signal at the correct time and for the correct duration is generated, the complex \acp{FFT} are then summed.

%%%%%%%%%%%%%%%
\subsection{Spectrogram injections}
%%%%%%%%%%%%%

To inject into a spectrogram the power spectrum of the signal will need to be simulated. In our injection we do not have access to a time-series, therefore, we do not simulate the signal in the same way, rather we use the signals estimated \ac{SNR}

It can be shown that the \ac{PSD} of Gaussian noise with zero mean and unit variance is a $\chi^2$ distribution with 2 degrees of freedom. Therefore, if we want to generate a spectrogram for Gaussian noise, we just generate a two dimensional array of values distributed as $\chi^2$ with two degrees of freedom.
Assuming that there is some sinusoidal signal with a given \ac{SNR} within a Gaussian noise time-series with zero mean and unit variance, the \ac{FFT} power in a particular frequency bin can be estimated using a non-central $\chi^2$ distribution with 2 degrees of freedom, where the non centrality parameter is the square of the \ac{SNR}. 
To calculate the \ac{SNR} in a given frequency bin the equation in \citep{prix2007SearchContinuous} for optimal \ac{SNR} was used,
\begin{equation}
    \rho(0)^2 = \frac{1}{2}h_0^2 T S^{-1} \left[ \alpha_1 A + \alpha_2 B + \alpha_3 C \right],
\end{equation}
where $h_0$ is the \ac{GW} amplitude, $T$ is the total observing time is seconds, $S^{-1}$ is the mean \ac{PSD} noise floor. The values of $\alpha$ are then defined by,
\begin{equation}
\begin{split}
\alpha_1 &= (\mathcal{A}^1)^2 + (\mathcal{A}^3)^2\\
\alpha_2 &= (\mathcal{A}^2)^2 + (\mathcal{A}^4)^2 \\
\alpha_3 &= \mathcal{A}^1\mathcal{A}^2 + \mathcal{A}^3\mathcal{A}^4 \\
\end{split}
\end{equation}
<br/>
where,
\begin{equation}
\begin{split}
\mathcal{A}^1 &= A_{+}\cos(2\psi_0)\cos(2\phi) - A_{\times}\sin(2\psi_0)\sin(2\phi) \\
\mathcal{A}^2 &= A_{+}\cos(2\psi_0)\sin(2\phi) + A_{\times}\sin(2\psi_0)\cos(2\phi) \\
\mathcal{A}^3 &= A_{+}\sin(2\psi_0)\cos(2\phi) - A_{\times}\cos(2\psi_0)\sin(2\phi) \\
\mathcal{A}^4 &= A_{+}\sin(2\psi_0)\sin(2\phi) + A_{\times}\cos(2\psi_0)\cos(2\phi) 
\end{split}
\end{equation}
The signals frequency varies with time and will not always be located at the center of a frequency bin, therefore, when taking the \ac{FFT} some of the power is spread over multiple frequency bins. 
In our injections into the power spectrum we need to account for this effect. 
For a given frequency bin width 


%%%%%%%
%%%%%%
\subsection{\label{intro:search:signals} Signals in data}
%%%%%%%
%%%%%%

The data recorded from a detector, $x(t)$, will include the signal model described in Eq.~\ref{sigmod} above, but it will be buried in the noise of the detector. 
If we assume the noise is Gaussian distributed and the noise and signal add linearly, then,  
\begin{equation}
\label{signalinnoise}
x(t) = n(t) + h(t; \mathcal{A},{\boldsymbol \lambda}) ,
\end{equation}
where $n(t)$ is the noise, $h(t)$ is the signal and $\mathcal{A}$ and ${\boldsymbol \lambda}$ refer to the amplitude and Doppler parameters respectively. 
The optimal signal to noise ratio (SNR) squared of this signal is defined as the scalar product of the signal with itself,
\begin{equation}
\rho^2(0) = ({\bf h} \mid {\bf h}) = \sum_X(h^X \mid h^X),
\end{equation}
where if there is more than one detector the SNR squared for each detector $X$ can be summed \citep{prix2007SearchContinuous}. 
The scalar product of two time series, $x(t)$ and $y(t)$, is defined by,

\begin{equation}
\label{intro:search:signals:scalarproduct}
(x \mid y) = 4 \Re \int_{0}^{\infty} \frac{\tilde{x}(f)\tilde{y}^{*}(f)}{S_n(f)}df,
\end{equation}
where $\tilde{x}(f)$ is the Fourier transform of $x(t)$, $\tilde{y}^{*}(f)$ is the complex conjugate of the Fourier transform of $y(t)$ and $S_n(f)$ is the single sided noise power spectral density \citep{prix2007SearchContinuous}.
