\chapter{\label{appb}Nested sampling derivation}

From the definition of the expectation value of a function $g(X)$ we
have
\begin{equation}
  \label{appb:bayes:nested:exectationg}
  E[ g(X) ] = \int g(x) f_X(x) dx,
\end{equation}
where $f_X(x)$ is the probability distribution of a random variable
$X$.
From this we can see that the evidence can be defined as
\begin{equation}
  \label{appb:bayes:nested:evidence_expec}
  Z = p(\bm{d} \mid I) = \int \mathcal{L}(\theta) \pi(\theta) d\theta
  = E [ \mathcal{L}(\theta)],
\end{equation}
where the prior $p(\theta \mid I) = \pi(\theta)$ is the probability
distribution of $\theta$ and the likelihood is $p(\bm{d} \mid \theta,
I) = \mathcal{L}(\theta)$.
From the definition of the expectation value of random variable $X$ we
can write
\begin{equation}
  \label{appb:bayes:nested:exectationx}
  E[X] = \int_0^{\infty} P(X > \lambda) d\lambda = \int_0^{\infty}
  \int_{\lambda}^{\infty} f_X(x) dx d\lambda,
\end{equation}
This can then be applied to
Eq.~\ref{appb:bayes:nested:evidence_expec} such that the Evidence
can be written as
\begin{equation}
  \label{appb:bayes:nested:evidence_cumul}
  Z = E[\mathcal{L}(\theta)] = \int_0^{\infty}  P(\mathcal{L}(\theta)
  > \lambda) d\lambda = \int_0^{\infty}  X(\lambda) d\lambda ,
\end{equation}
where we can define $X = P(\mathcal{L}(\theta) > \lambda)$ as the
prior mass
\begin{equation}
  \label{appb:bayes:nested:priormass}
  X(\lambda) = \int_{\mathcal{L}(\theta) > \lambda} \pi(\theta)
  d\theta,
\end{equation}
which is the amount of the prior where $\mathcal{L}(\theta) >
\lambda$.
As the prior mass is an integral of a probability distribution, we
know that it has a minimum value of zero and a maximum value of 1,
therefore, we can rewrite
Eq.~\ref{appb:bayes:nested:evidence_cumul} as
\begin{equation}
  \label{appb:bayes:nested:evidence}
  Z = p(\bm{d} \mid I) = \int \mathcal{L}(\theta) \pi(\theta) d\theta
  = \int_0^1 \mathcal{L}(X) dX,
\end{equation}
where the function $\mathcal{L}(X)$ is the value of the likelihood
such that $P(\mathcal{L}(X) > \lambda) = X$.
This then mean that we have a one dimensional integral over the prior
mass $X$ which has a range between zero and one.