\chapter{\label{searchcw}Searching for continuous gravitational waves}

Continuous gravitational waves have particular challenges when it comes to their detection.
Whilst I have described the potential sources of the signal and its signal type in Sec.~\ref{intro:signals:cw}, I will go into more detail on the signal and search algorithms here.



\section{\label{searchcw:model}Continuous signal model}

The model of a \ac{GW} signal from a pulsar is relatively simple, it is a quasi-sinusoidal signal. This means that the signal is a sinusoid with a slowly varying frequency. One reason for the slow variance in the frequency is due to the energy loss to \ac{GW} as the pulsar spins down.
Here the signal is modelled to originate from an isolated triaxial neutron star rotating around a principal axis. 
The parameters of each pulsar can be split into two sections: the Doppler components ($\alpha,\delta,{\bm f}$) and its amplitude components ($\psi,\phi_0, \iota, h_0, \theta$). This ignores any orbital parameters which would be present if the star was in a binary systems and higher order frequency derivatives.
They are defined as follows: the sky positions $\alpha$ and $delta$ refer to the right ascension and declination. 
${\bm f}$ refers to the source frequency and its derivatives. 
$\psi$ and $\phi_0$ and $h_0 $ are the \ac{GW} polarisation, initial phase and amplitude respectively. 
$\iota$ is the inclination angle which is how much the source is tilted relative to the observer. 
$\theta$ is the `wobble angle' or the angle between the rotation axis and the symmetry axis of the neutron star.

The definition of the \ac{GW} from a neutron star here follows that in \citep{schutz1998DataAnalysis,riles2017RecentSearches,dupuis2005BayesianEstimation}. The amplitude of the \ac{GW} can be defined as,
\begin{equation}
\label{intro:cw:ht}
h(t) = F_+(t)h_{+}(t) +F_{\times}(t)h_{\times}(t),
\end{equation}
where $h_{+},h_{\times}$ are the plus and cross polarisations functions as in Eq.\ref{intro:gravwave:amp} \joe{check reference} and $F_{+},F_{\times}$ are the antenna pattern functions to the two polarisations.
These are defined by,
\begin{equation}
\label{intro:cw:amplitudes}
    \begin{split}
        h_{+}(t) &=  h_0 \frac{1 + \cos^2{(\iota)}}{2}\cos{\left(\Phi(t)\right)} \\
        h_{\times}(t) &= h_0  \cos{(\iota)} \sin{\left( \Phi(t)\right) } \\
    \end{split}
\end{equation}
The plus and cross polarised components then depend on the \ac{GW} amplitude $h_0$, the inclination angle of the source $\iota$ and the phase evolution of the \ac{GW}. Here I have chosen to assume a small wobble angle $\theta$, however, this is included in \citep{schutz1998DataAnalysis}. The phase of the wave $\Phi(t_{{\rm SSB}})$ at the \ac{SSB} can be defined as,
\begin{equation}
\label{searchcw:model:phase}
    \Phi(t_{{\rm SSB}}) = \phi_0 + 2\pi\left[ f(t_{{\rm SSB}} - t_0) + \frac{1}{2} \dot{f} (t_{{\rm SSB}} - t_0)^2 + .....\right] .
\end{equation}
This consists of an initial phase $\phi_0$ which is the phase at time $t_0$, the frequency of the signal $f$ and its derivative ${\dot{f}}$ at time $t_0$. Here we show the phase to second order, however, this can be easily extended if necessary. 
The time at the \ac{SSB} $t_{{\rm SSB}}$ can be transformed to the time $t$ at the detector by,
\begin{equation}
t_{{\rm SSB}} = t - \frac{\mathbf{r}_d \cdot \mathbf{k}}{c} + \delta_t.
\end{equation}
Here $\mathbf{r}_d$ is the position of the detector with reference to the \ac{SSB}, $\mathbf{k}$ is a unit vector in the direction of the source. This essentially takes into account the Doppler shift of the signal due to the movement of the detector, i.e. as the earth rotates and orbits the sun. $c$ is the speed of light and $\delta t$ is extra corrections from the Einstein, Binary and Shapiro delay \citep{}.
The amplitudes $h_0$ in Eq.~\ref{intro:cw:amplitudes} are defined by,
\begin{equation}
    h_0 = \frac{16 \pi^2 G}{c^4} \frac{\epsilon I f^2}{r},
\end{equation}
where $G$ is the gravitational constant, $c$ is the speed of light, $\epsilon$ is the ellipticity of the star, $f$ is the sum of the frequency of rotation of the star and the frequency of precession, $r$ is the distance to the star and $I_{zz}$ is the moment of inertia with respect to the rotation axis $z$.
The ellipticity of the star $\epsilon$ is a measure of the distortion of the star around its rotation axis and is defined by,
\begin{equation}
    \epsilon = \frac{I_{xx} - I_{yy}}{I_{zz}},
\end{equation}
where $I_{xx}, I_{yy}$ and $I_{zz}$ are the moments of inertia for each axis.

In Eq.~\ref{intro:cw:ht}, $F_+(t)$ and $F_{\times}(t)$ are the antenna pattern functions of the detector. 
These describe how sensitive a detector is to a particular location on the sky at any given time. 
The amplitude of the signal will vary dependent on the orientation and location of the detector relative to the source.
This is described in Sec.~\ref{intro:detector} and the response to sky location is shown in Fig.~\ref{intro:detectors:response}.
These components are defined in \citep{schutz1998DataAnalysis} as,
\begin{equation}
\label{intro:cw:antenna}
\begin{split}
F_{+}(t) &= \sin{\zeta}[a(t)\cos{(2\psi)} + b(t)\sin{(2\psi)}], \\
F_{\times}(t) &= \sin{\zeta}[b(t) \cos{(2\psi)} - a(t)\sin{(2\psi)}],
\end{split}
\end{equation}
where $\zeta$ is the angle between the arms of the detectors, $\psi$ is the polarisation angle of the \ac{GW} and $a(t)$ and $b(t)$ are defined in \citep{schutz1998DataAnalysis} and relate the sky location to the orientation of the detector at a given time. 
A full derivation of this can be found in \citep{schutz1998DataAnalysis} where each of these terms are expanded.

Eqs.~\ref{intro:cw:ht} - \ref{intro:cw:antenna} then describe the amplitude and phase evolution of a signal at a given detector location and orientation.


%%%%%%%%%%%55
%%%%%%%%%%%%%
\section{\label{searchcw:search} Continuous wave searches}
%%%%%%%%%%%%%%%%%%
%%%%%%%%%%%%%%%%%%

There are many different methods to search for continuous gravitational waves.
They can be split into three general categories: Targeted searches, Directed searches and All-sky searches.
The main difference between these different categories is the amount which is known about the source prior to the search.
For targeted searches the sky position $(\alpha,\delta)$ and rotation frequency are known from electromagnetic observations, i.e. X-ray, radio or $\gamma$-ray.
Directed searches have information on the sky position $(\alpha,\delta)$ but not the rotation frequency.
For all-sky searches, there is no prior knowledge of the pulsar, therefore, is a search for unknown pulsars.
In general the searches in each of these categories use two distict techniques: Fully coherent searches and Semi-coherent searches

%%%%%%
%%%%%%
\subsection{\label{searchcw:search:coherent}Fully coherent}
%%%%%%
%%%%%%

A Fully coherent search generaly uses a pre generated waveform which follows the model described in Sec.~\ref{searchcw:model}. 
This contains all the phase information of the signal.
The set of parameters which genereated the waveform which `matches' the best can then be considered as the optimum set of parameters given the data.
This is known as a matched filter \citep{} and is used in \ac{CW} searces in \citep{dupuis2005BayesianEstimation,}.

The matched filter maximises the signal to noise ratio for a given filter, in this case the filter is our \ac{CW} model. 
The matched filter used for \ac{CW} models is defined in \citep{prix2007SearchContinuous} and it titled the $\mathcal{F}$-statistic. 
This essentially maximises a likelihood with respect to the parameters.
If one assumes that the noise $n$ is Gaussian and zero mean, the data $x$ can be written as,
\begin{equation}
		x(t) = n(t) + h(t).
\end{equation}
The likelihood can then be written as,
\begin{equation}
		\log \Lambda = \left( x \mid h \right) - \frac{1}{2} \left( h \mid h\right) 
\end{equation}
where the product $(g \mid g)$ is defined as,
\begin{equation}
		\left( x \mid y \right) = 4 \mathcal{R} \int^{\infty}_{-\infty}  \frac{\tilde{x}^{{\rm X}}(f) \tilde{y}^{{\rm X *}}(f)  }{S^{X}(f)} \mathrm{d}f.
\end{equation}
This is fulle expanded into the $\mathcal{F}$-statistic in \citep{schutz1998DataAnalysis}, however, it is essentially this likelihood function which is maximised. 


Targeted searches look for a specific pulsar which has been observed in the electromagnetic spectrum.
These observations give information such as the sky position and the frequency evolution of the source.
Using knowledge of the earths position around the sun, which is well known, one can use the accurate sky position and frequency of a known source to find its phase evolution in Eq.~\ref{searchcw:model:phase}.
This mean that for this type of search one can maximise the likelihood with respect to the parameters  $h_0, \phi_0, \iota$ and $\psi$.
Another method which uses templates is described in \citep{dupuis2005BayesianEstimation}, this uses a Bayesian approach.

This type of search can take long periods of time. 
This is due both to the size of the parameter space and the amount of data which needs to be searched.
\ac{CW} searches need long observation times in order to accumulate the required \ac{SNR} for detection.
Therefore, most searches use data from an entire \ac{LIGO} observing run which can last for $\mathcal{O}(1)$ years.
Given that the sampling rate for the \ac{GW} channel is $16$ kHz usually downsampled to $\sim 4$ kHz, the quantity of data is large. 


Whilst the fully coherent matched filter searches have methods to reduce the computational time for known sources, in all-sky and directed searches, this type of search is no feasible. 
This is becasue all-sky and directed searches have a wider parameter space, therefore, enough templates need to be made to sufficiently cover the large parameter space. 
This task quickly becomes impossible for coherent matched filtering for an entire observing run due to the amount of time needed. This problem led to the development of semi-coherent searches which will be introduced in the next section. 


%%%%%%
%%%%%%
\subsection{\label{searchcw:search:semi}Semi coherent}
%%%%%%
%%%%%%

Semi-coherent searches offered a solution to searching over large parameters spaces and large amounts of data. 
As is directed and all-sky searches the phase evolution of the source is not known, one cannot use a coherent search for the entire observing run.
It may however, be possible to approximately describe the phase for a shorter length of time known as the coherence time, $T_{coh}$.
The general idea of a semi-coherent search is to break the data-set into smaller section which each can be analysed coherently.
The coherent analysis can use the matched filter as described in Sec.~\ref{searchcw:search:coherent} or another method such as a Fourier transform.
The results from each of these individual sections can the be combined incoherently using various methods which will be summarised later. 
This method can greatly reduce the time taken for the analysis depending on the coherence length, however, will always come with some loss in sensitivity. 

There are many different types of semi-coherent search which use various methods to incoherently combine the coherently analysed results. 
I will summarise some of these searches below, some of these searches were summarised and compared in \citep{walsh2016ComparisonMethods}.

\begin{description}
	
	\item[Stack-slide] Stack uses a set of Fourier transforms of the data known as \acp{SFT}, specifically it uses the power spectrum of these. Each of the separate Fourier transforms (segments) is shifted up or down relative to the others to account for the doppler modulation of the source. The power from each can then be stacked. More explanation of this can be found in \citep{}  
	
	\item[Hough] The Hough transform is based on the stack-slide algorithm. The main difference is that the detectionn statistic for each segment is assigned a weight of 0 or 1 depending if it crossed a detection threshold. The Hough transform can the create a `Hough map' which gives a view of the data in parameter space. This approach is explained in greated detail in \citep{}. 
	This method has been applied in two main ways known as Sky Hough and Frequency Hough
	
	\item[Einstein@Home] Einstein at 
	
	\item[Time domain $\mathcal{F}$-statistic]
	
	\item[Powerflux]
	
	
\end{description}



