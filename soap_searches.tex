%!TEX root = main.tex
\chapter{SOAP search}

%%%%%%%%%%%
%%%%%%%%%%
\section{Pipeline}
%%%%%%%%%%%
%%%%%%%%%%

The \ac{SOAP} pipeline itself contains two key elements: the Viterbi search described in Sec.~\ref{viterbi} and the \ac{CNN} part of the search described in Sec.~\ref{cnn}. However, there are many other steps within the search which have to take place to allow the key search components, these are summarised in Fig.~\ref{pipeline:flow} and are described below.

\begin{description}
    \item[1. \acp{SFT}] Generate 1800s long \acp{SFT} from detector time-series data. 

    \item[2. Narrowbanding] To improve the efficiency of the search the
    narroband \acp{SFT} are split into $2.1$ Hz wide bands every $2$ Hz,
    i.e. 100.0-102.1, 102.0-104.1 etc.
    
    \item[3. Normalising] The spectrogram are then normalised to their running median such that they have a mean of 1. Each spectrogram is then multiplied by 2 such that they are approximately $\chi^{2}$ distributed.
    
    \item[4a. Search data] For search data the bands are split up into 0.1 Hz wide sub-bands overlapping by 0.05 Hz.

    \item[4b. Training data generation] To generate training data the
    process is the same as in Sec.~\ref{data}. Where the
    data is split into 0.1 Hz wide bands which are not
    overlapping. Each of these sub-bands is the `augmented' as in Sec.~\ref{data:augmentation}. Injections are then made into each of these bands with \acp{SNR} in the range 50-150. The bands are then split into `odd' and `even' categories.

    \item[4c. Test data generation] Here the narrow-banded
    spectrogram are further split into 0.1 Hz wide sub-bands which
    are overlapping by 0.05 Hz. The overlapping sub-bands mean that an
    astrophysical signal should be fully contained within one
    sub-band. Then signals following parameters in Tab.~\ref{data:injections:table} are injected in to 50\% of the sub-bands with \acp{SNR} in the range 20-200.

    \item[5. Summing spectrogram] As in \cite{Bayley2019GeneralizedSignals} the spectrogram are summed over one day, i.e. every 48 time segments of the spectrogram.
     
    \item[6. Generate lookup tables and Run Viterbi search] Before the Viterbi search is run, the line-aware statistic lookup tables need to be generated as in \cite{Bayley2019GeneralizedSignals}. Then for each of the 3 data sets the Viterbi search is run. 
     
    \item[7. Down-sample data] At this stage there are four elements which are saved separately for each data-set. The two time-frequency maps, the Viterbi maps and the Viterbi statistic. The time-frequency maps and the Viterbi maps are down-sampled to a size of (156x89) using interpolation from scikit-image's resize \cite{vanderWalt2014Scikit-image:Python}. This size was chosen based on the S6 \ac{MDC} data-set, where this is 1/3 the length in time and 1/2 the width in frequency of the summed spectrograms.


    \item[8. Train Networks] The down-sampled training data is then used to train 12 separate \acp{CNN}. Half of these networks are trained on the `odd' bands and the other half are trained on the `even' bands. 

    \item[9a. Run search on real data] The trained network from 7. is then used to classify each sub-band in the search data, this returns a statistic in $[0,1]$ where 1 represents the probability of a signal. 
    
    \item[9c. Run search on test data] The trained network from 7. is then used to classify each sub-band in the test injected data, this returns a statistic in $[0,1]$ where 1 represents the probability of a signal. 
    
    \item[10a. Signal candidates] The signals which have a statistic in the top 1\% are taken for a followup investigation. This can be another search, or just a look `by-eye'.
    
    \item[10c. Efficiency curves] The statistics can be plotted against \ac{SNR} to see how the network classified signals with the \ac{SNR} of the injection. Then the efficiency curves can be generated, this is described in further detail in Sec.~\ref{sensitivity}.
    

  
\end{description}



%%%%%%%%%%%%
%%%%%%%%%%%%%%
\section{Injections}
%%%%%%%%%%%%%%
%%%%%%%%%%%%%%%%
In this section I outline how we inject a \ac{CW} signal into data. This can generally be done in two different ways: simulating a signal in the time domain and injecting into time domain noise or simulating the power spectrum of a signal and injecting the signal into a \ac{PSD}.
%%%%%%%%%%%
\subsection{CW Signal}
%%%%%%%%%%%
This section has been covered in Sec.~\ref{intro:cw:signal}, The CW signal in generated by using a lalsuite package where ....

%%%%%%%%%%%%
\subsection{Time series and complex \ac{FFT} injections}
%%%%%%%%%%%%
Injections into time-series data is relatively simple. Given a set of parameters for the source the signal can be generated in the time-series, this is then just summed with the time-series which it is injected into. Similarly with the \ac{FFT}, the time-series of the signal at the correct time and for the correct duration is generated, the complex \acp{FFT} are then summed.

%%%%%%%%%%%%%%%
\subsection{Spectrogram injections}
%%%%%%%%%%%%%

To inject into a spectrogram the power spectrum of the signal will need to be simulated. In our injection we do not have access to a time-series, therefore, we do not simulate the signal in the same way, rather we use the signals estimated \ac{SNR}

It can be shown that the \ac{PSD} of Gaussian noise with zero mean and unit variance is a $\chi^2$ distribution with 2 degrees of freedom. Therefore, if we want to generate a spectrogram for Gaussian noise, we just generate a two dimensional array of values distributed as $\chi^2$ with two degrees of freedom.
Assuming that there is some sinusoidal signal with a given \ac{SNR} within a Gaussian noise time-series with zero mean and unit variance, the \ac{FFT} power in a particular frequency bin can be estimated using a non-central $\chi^2$ distribution with 2 degrees of freedom, where the non centrality parameter is the square of the \ac{SNR}. 

To calculate the \ac{SNR} in a given frequency bin the equation in \cite{Prix2007} for optimal \ac{SNR} was used,
\begin{equation}
    \rho(0)^2 = \frac{1}{2}h_0^2 T S^{-1} \left[ \alpha_1 A + \alpha_2 B + \alpha_3 C \right],
\end{equation}
where $h_0$ is the \ac{GW} amplitude, $T$ is the total observing time is seconds, $S^{-1}$ is the mean \ac{PSD} noise floor. The values of $\alpha$ are then defined by,
\begin{equation}
\begin{split}
\alpha_1 &= (\mathcal{A}^1)^2 + (\mathcal{A}^3)^2\\
\alpha_2 &= (\mathcal{A}^2)^2 + (\mathcal{A}^4)^2 \\
\alpha_3 &= \mathcal{A}^1\mathcal{A}^2 + \mathcal{A}^3\mathcal{A}^4 \\
\end{split}
\end{equation}
<br/>
where,
\begin{equation}
\begin{split}
\mathcal{A}^1 &= A_{+}\cos(2\psi_0)\cos(2\phi) - A_{\times}\sin(2\psi_0)\sin(2\phi) \\
\mathcal{A}^2 &= A_{+}\cos(2\psi_0)\sin(2\phi) + A_{\times}\sin(2\psi_0)\cos(2\phi) \\
\mathcal{A}^3 &= A_{+}\sin(2\psi_0)\cos(2\phi) - A_{\times}\cos(2\psi_0)\sin(2\phi) \\
\mathcal{A}^4 &= A_{+}\sin(2\psi_0)\sin(2\phi) + A_{\times}\cos(2\psi_0)\cos(2\phi) 
\end{split}
\end{equation}
The signals frequency varies with time and will not always be located at the center of a frequency bin, therefore, when taking the \ac{FFT} some of the power is spread over multiple frequency bins. 
In our injections into the power spectrum we need to account for this effect. 
For a given frequency bin width 

%%%%%%%%%
%%%%%%%%
\section{Viterbi lookup tables}
%%%%%%%%%%
%%%%%%%%%

In Sec.~\ref{viterbi:lineaware} the line aware statistic to be used with the \ac{SOAP} search was described. This part of the search is pre-calculated into lookup tables in an effort to decrease the total computational time of the search. 

%%%%%%%%%%%%%
%%%%%%%%%%%%%%%
\section{Gaussian Noise}
%%%%%%%%%%%%%%%%
%%%%%%%%%%%%%

%%%%%%%%%%%
%%%%%%%%%%%%
\section{S6}

\subsection{S6 MDC}

\subsection{S6}

\section{O1}

\subsection{Astrophysical}

\subsection{Instrumental Lines}

\section{O2}

\subsection{Astrophysical}

\subsection{Instrumental Lines}

\section{O3}

\subsection{Astrophysical}

\subsection{Instrumental Lines}
