\chapter{\label{searchcw}Searching for continuous gravitational waves}

Continuous gravitational waves have particular challenges when it comes to their detection.
Whilst I have described the potential sources of the signal and its signal type in Sec.~\ref{intro:signals:cw}, I will go into more detail on the signal and search algorithms here.



\subsection{\label{intro:search:model}Continuous signal model}

The model of a \ac{GW} signal from a pulsar is relatively simple, it is a quasi-sinusoidal signal. This means that the signal is a sinusoid with a slowly varying frequency. One reason for the slow variance in the frequency is due to the energy loss to \ac{GW} as the pulsar spins down.
Here the signal is modelled to originate from an isolated triaxial neutron star rotating around a principal axis. 
The parameters of each pulsar can be split into two sections: the Doppler components ($\alpha,\delta,{\bm f}$) and its amplitude components ($\psi,\phi_0, \iota, h_0, \theta$). This ignores any orbital parameters which would be present if the star was in a binary systems and higher order frequency derivatives.
They are defined as follows: the sky positions $\alpha$ and $delta$ refer to the right ascension and declination. 
${\bm f}$ refers to the source frequency and its derivatives. 
$\psi$ and $\phi_0$ and $h_0 $ are the \ac{GW} polarisation, initial phase and amplitude respectively. 
$\iota$ is the inclination angle which is how much the source is tilted relative to the observer. 
$\theta$ is the `wobble angle' or the angle between the rotation axis and the symmetry axis of the neutron star.

The definition of the \ac{GW} from a neutron star here follows that in \citep{Schutz1998DataDetection}. The amplitude of the \ac{GW} can be defined as,
\begin{equation}
\label{intro:cw:ht}
h(t) = F_+(t)(h_{1+}(t) + h_{2+}(t)) +F_{\times}(t)(h_{1\times}(t) + h_{2\times}(t)),
\end{equation}
where $h_{1+}$ and $h_{2+}$ are the two components of the plus polarisation and $h_{1\times}$ and $h_{2\times}$ are the two components of the cross polarisation functions.
These are defined by,
\begin{equation}
\label{intro:cw:amplitudes}
    \begin{split}
        h_{1+}(t) &= \frac{1}{8} h_0 \sin{2\theta} \sin{2\iota}\cos{\Phi(t)} \\
        h_{2+}(t) &= \frac{1}{2} h_0 \sin^2{\theta} \left( 1 + \cos^2{\iota}\right) \cos{2\Phi(t)} \\
        h_{1\times}(t) &= \frac{1}{4} h_0 \sin{2\theta} \sin{\iota} \sin{\Phi(t)} \\
        h_{2\times}(t) &= h_0 \sin^2{\theta} \cos{\iota} \sin{2\Phi(t)}.
    \end{split}
\end{equation}
The plus and cross polarised components then depend on the \ac{GW} amplitude $h_0$, the wobble angle $\theta$ and the inclination angle of the source $\iota$. The phase of the wave $\Phi(t)$ can be defined as,
\begin{equation}
    \Phi(t) = \phi_0 + \phi_{f}(t,f) + \phi_{{\rm sky}}(t,\alpha,\delta).
\end{equation}
This consists of an initial phase $\phi_0$, a component of the phase which describes how the frequency of the source changes with time $\phi_f(t)$ and a component which accounts for the Doppler shift of the detectors as the earth rotates and orbits the sun $\phi_{{\rm sky}}(t,\alpha,\delta)$.
A full derivation of this can be found in \citep{Schutz1998DataDetection} where each of these terms are expanded.
The amplitudes $h_0$ in Eq.~\ref{intro:cw:amplitudes} are defined by,
\begin{equation}
    h_0 = \frac{16 \pi^2 G}{c^4} \frac{\epsilon I f^2}{r},
\end{equation}
where $G$ is the gravitational constant, $c$ is the speed of light, $\epsilon$ is the ellipticity of the star, $f$ is the sum of the frequency of rotation of the star and the frequency of precession, $r$ is the distance to the star and $I_{zz}$ is the moment of inertia with respect to the rotation axis $z$.
The ellipticity of the star $\epsilon$ is a measure of the distortion of the star around its rotation axis and is defined by,
\begin{equation}
    \epsilon = \frac{I_{xx} - I_{yy}}{I_{zz}},
\end{equation}
where $I_{xx}, I_{yy}$ and $I_{zz}$ are the moments of inertia for each axis.

In Eq.~\ref{intro:cw:ht}, $F_+(t)$ and $F_{\times}(t)$ are the antenna pattern functions of the detector. 
These describe how sensitive a detector is to a particular location on the sky at any given time. 
The amplitude of the signal will vary dependent on the orientation and location of the detector relative to the source.
This is described in Sec.~\ref{intro:detector} and the response to sky location is shown in Fig.~\ref{intro:detectors:response}.
These components are defined in \citep{Schutz1998DataDetection} as,
\begin{equation}
\label{intro:cw:antenna}
\begin{split}
F_{+}(t) &= \sin{\zeta}[a(t)\cos{(2\psi)} + b(t)\sin{(2\psi)}], \\
F_{\times}(t) &= \sin{\zeta}[b(t) \cos{(2\psi)} - a(t)\sin{(2\psi)}],
\end{split}
\end{equation}
where $\zeta$ is the angle between the arms of the detectors, $\psi$ is the polarisation angle of the \ac{GW} and $a(t)$ and $b(t)$ are defined in \citep{Schutz1998DataDetection} and relate the sky location to the orientation of the detector at a given time. 

Eqs.~\ref{intro:cw:ht} - \ref{intro:cw:antenna} then describe the amplitude and phase evolution of a signal at a given detector location and orientation.


%%%%%%%
%%%%%%
\subsection{\label{intro:search:signals} Signals in data}
%%%%%%%
%%%%%%

The data recorded from a detector, $x(t)$, will include the signal model described in Eq.~\ref{sigmod} above, but it will be buried in the noise of the detector. 
If we assume the noise is Gaussian distributed and the noise and signal add linearly, then,  
\begin{equation}
\label{signalinnoise}
x(t) = n(t) + h(t; \mathcal{A},{\boldsymbol \lambda}) ,
\end{equation}
where $n(t)$ is the noise, $h(t)$ is the signal and $\mathcal{A}$ and ${\boldsymbol \lambda}$ refer to the amplitude and Doppler parameters respectively. 
The optimal signal to noise ratio (SNR) squared of this signal is defined as the scalar product of the signal with itself,
\begin{equation}
\rho^2(0) = ({\bf h} \mid {\bf h}) = \sum_X(h^X \mid h^X),
\end{equation}
where if there is more than one detector the SNR squared for each detector $X$ can be summed \citep{Prix2007}. 
The scalar product of two time series, $x(t)$ and $y(t)$, is defined by,

\begin{equation}
\label{intro:search:signals:scalarproduct}
(x \mid y) = 4 \Re \int_{0}^{\infty} \frac{\tilde{x}(f)\tilde{y}^{*}(f)}{S_n(f)}df,
\end{equation}
where $\tilde{x}(f)$ is the Fourier transform of $x(t)$, $\tilde{y}^{*}(f)$ is the complex conjugate of the Fourier transform of $y(t)$ and $S_n(f)$ is the single sided noise power spectral density \citep{Prix2007}.

%%%%%%
%%%%%%
\subsection{\label{intro:search:coherent}Fully-coherent searches}
%%%%%%
%%%%%%

Fully coherent searches are currently the most sensitive searches for continuous sources of gravitational waves. These searches are based on matched filtering which coherently correlate pre-generated waveforms with the data, this is used in \citep{Dupuis2005,Jaranowski1998,}.
In a simple form the matched filter filters the data $h$ with a template $w$ using the inner product defined in Eq.~\ref{intro:search:signals:scalarproduct}. 

Continuous wave signals need a large amount of integration time, $\mathcal{O}(\rm{years})$, for the signal to become clear within the data, therefore, given that the \ac{LIGO} detectors sample at 16 kHz, the amount of data to search over is large. 
Performing the coherent matched filter on this data can take a large amount of time, therefore, as the majority of sources which are searched for have a narrow bandwidth, many searches use techniques to reduce the amount of data. 
For example, in \citep{Dupuis2005} the data is heterodyned (mixed with some local oscillator), low passed filtered and then down-sampled. This reduced the amount of data to be searched over while not losing any source information. 

Whilst the fully coherent matched filter searches have methods to reduce the computational time for known sources, in all sky searches no parameters of the source are known, therefore, enough templates need to be made to sufficiently cover the large parameter space. 
This task quickly becomes impossible for coherent matched filtering for an entire observing run due to the amount of time needed. This problem led to the development of semi-coherent searches which will be introduced in the next section. 

%%%%%%
%%%%%%
\subsection{\label{intro:search:semicoherent}Semi-coherent searches}
%%%%%%
%%%%%%

Semi-coherent searches offered a solution to searching over large parameters spaces and large amounts of data. 
The data is split into smaller segments and the analysis is run separately on each of those segments, then each result is combined incoherently. 
This can greatly reduce the time taken for the analysis depending on the segment length, however, will always come with some loss in sensitivity. 

There are many different types of semi-coherent search which use various methods to incoherently combine the coherently analysed results. Many of these use 1800s (30 mins) \acp{SFT} as the input to their search as they can be calculated efficiently. 


