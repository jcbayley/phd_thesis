\chapter{Abstract}

The field of gravitational wave astronomy is still in its early stages, with
detections of compact binary coalescences numbering $\sim
60$~\chris{techncially we haven't claimed that ythe O3 events are indeed
detections. We still only have 11 from O1 and O2 plus the recent BNS and
GW190412. All other events are "candidates".}. Another possible source of
gravitational waves is rapidly rotating neutron stars which have some asymmetry
around their rotation axis~\chris{ambiguous - are they possible source AND they
have asymmetry or are they possible sources BECAUSE they have asymmetry?}.
These are predicted to emit long duration quasi-sinusoidal signals known as
continuous gravitational waves.

% SOAP chapter
All-sky and wide parameter space searches for continuous gravitational waves
are generally template-matching schemes which test a bank of signal waveforms
against data from a gravitational wave detector.  Such searches can offer
optimal~\chris{I'd be careful here. There is no optimality at play. Really,
it's a case of having an algorithm and then tuning its own sensitivity at a
fixed computational cost. Your algorithm is basically as sensitive as others
but uses a tiny fraction of the cost so you can't say that any of the
approaches are "optimal".} sensitivity for a given computing cost and signal
model, but are highly-tuned to specific signal types and are computationally
expensive, even for semi-coherent~\chris{try to keep techncial jargon out of
the abstract} searches. We have developed a search method based on the Viterbi
algorithm which is model-agnostic and has a computational cost several orders
of magnitude lower than template methods,~\chris{and} with a modest reduction
in sensitivity~\chris{is this true? I thought we were in the mix compared to
all the semi-coherent approaches.}. In particular, this method can search for
signals which have an unknown frequency evolution. We test the algorithm on
three simulated and real data sets: gapless Gaussian noise, Gaussian noise with
gaps and real data from the final run of initial LIGO (S6). We show that at
95\% efficiency, with a 1\% false alarm rate, the algorithm has a depth
sensitivity~\chris{depth is a bit technical unless defined. It's relatively
simple to define so you might think about saying "depth sensitivity
$S_{h}(f)/h_0$" of $\ldots$. Careful to then also define the variables. It
might be a bit messy.} of $\sim 33$, $10$ and $13$\,Hz$^{-1/2}$ with
corresponding ~\chris{it's important to state that this is the optimal coherent
SNR and not the SNR of the viterbi statistic.} SNRs~\chris{careful with
undefined acronymns} of $\sim 60$, $72$ and $74$ in these datasets. We discuss
the use of this algorithm for detecting a wide range of quasi-monochromatic
gravitational wave signals and instrumental lines~\chris{jargon?}, and
demonstrate that it can also highlight~\chris{highlight is vague. What does
that mean?} shorter duration signals such as compact binary coalescences.


% Machine learning chapter
Many continuous gravitational wave searches are affected by instrumental lines
as the long duration narrowband nature of a line can appear to be very similar
to a real continuous~\chris{gravitational?} wave signal.  This has led
to~\chris{the development of} techniques to try and limit the effect of
instrumental lines, which mostly involve developing a statistic to penalise
signals that appear in only a single detector.  We have developed a method
using convolutional neural networks to reduce the impact of instrumental
artefacts on the SOAP~\chris{SOAP hasn't been defined.} search.  This has the
ability to identify features in each detectors spectrograms such that a
frequency band can be classified into a signal or noise class.  Using this
method we achieve a similar sensitivity to the SOAP search alone~\chris{make
this clearer to the reader. I know what you mean but it sounds like you've made
a change that has no effect}, however, this allows SOAP to be a fully
automated~\chris{the importance of automation is only important if the reader
knows about the manual line identification.} search which will return
candidates to be followed up.~\chris{this paragrpah can be made a lot clearer.}


% Parameter est chapter 
Once a continuous gravitational wave is detected, we would want to extract some
parameters associated with the source to help understand more about its
structure and evolution.  We describe a Bayesian method which extracts the sky
location, frequency and frequency derivative of a source associated with the
frequency evolution returned by the SOAP algorithm.  This has the aim of
limiting the size of the parameter space for a more sensitive fully coherent
follow up search.  We demonstrate a model which currently does not provide
valid estimates of the source parameters, however, with further investigation
we aim to develop this into a reliable analysis.~\chris{You maybe don't need to
be as bluntly negative and honest as you are here. I would advise slightly
expanding the description but also say that we present results and describe the
features and shortcomings of our approach. Or something like that. While we are
talimng about the PE, I would expect the results to be more presentable if you
take the 50\% confindence interval as opposed to the 95\%. Have you been bale
to do this yet?}

% Lines chapter
As mentioned above, we limit the effect of instrumental lines on the SOAP
search using machine learning, however we can also identify and mitigate these
lines separately before a search is run.  We demonstrate how we can use SOAP in
a simple configuration to identify instrumental lines.  We compare this method
to existing line identification tools used in the \gls{LIGO} collaboration, and
find that SOAP identifies many of the same lines as these methods as well as
some which do not appear on their line lists.~\chris{You could expand this a
bit but maybe more importtantly, since this is an abstract, you should make
some quantitative statements about the numbers/fractions of common lines found
as well as the numbers of new lines SOAP finds.}







