\chapter{Abstract}

The field of gravitational wave astronomy is still in its early stages, with
detections of compact binary coalescences numbering $\sim
12$ and the most recent observing run (O3) providing $\sim 50$ more candidates. Another possible source of
gravitational waves is rapidly rotating neutron stars which can emit gravitational waves if they have some asymmetry
around their rotation axis.
These are predicted to emit long duration quasi-sinusoidal signals known as
continuous gravitational waves.

% SOAP chapter
All-sky and wide parameter space searches for continuous gravitational waves
are generally template-matching schemes which test a bank of signal waveforms
against data from a gravitational wave detector.  Often these searches  are highly-tuned to specific signal types and are computationally
expensive. We have developed a search method (entitled SOAP) based on the Viterbi
algorithm which is model-agnostic and has a computational cost several orders
of magnitude lower than template methods and with a comparable sensitivity. 
In particular, this method can search for
signals which have an unknown frequency evolution. We test the algorithm on
three simulated and real data sets: gapless Gaussian noise, Gaussian noise with
gaps and real data from the final run of initial LIGO (S6). We show that at
95\% efficiency, with a 1\% false alarm rate, the algorithm achieves a sensitivity of $\sim 60,\, 72$ and $74$ in the optimal coherent signal to noise ratio in each of these datasets.
We discuss the use of this algorithm for detecting a wide range of quasi-monochromatic
gravitational wave signals and instrumental artefacts, and
demonstrate that it can also identify shorter duration signals such as compact binary coalescences.


% Machine learning chapter
Many continuous gravitational wave searches are affected by instrumental lines
as the long duration narrowband nature of a line can appear to be very similar
to a real continuous gravitational wave signal.  This has led
to the development of techniques to try and limit the effect of
instrumental lines, which mostly involve developing a statistic to penalise
signals that appear in only a single detector.  
Whilst these statistics limit the effect of instrumental lines, in the SOAP search described above, many lines still contaminate the statistics and have to be manually removed by investigating other search outputs.
We have developed a method using convolutional neural networks to reduce the impact of instrumental
artefacts on the SOAP search described above.  This has the
ability to identify features in each of the detectors spectrograms such that a
frequency band can be classified into a signal or noise class.  
This limits the amount of manual investigation of frequency bands and allowed the SOAP search to be fully automated without a reduction in the sensitivity.



% Parameter est chapter 
Once a continuous gravitational wave is detected, we would want to extract some
parameters associated with the source to help understand more about its
structure and evolution. We describe a Bayesian method which extracts the sky
location, frequency, frequency derivative and signal to noise ratio of a source associated with the
frequency evolution returned by the SOAP algorithm.  This has the aim of
limiting the size of the parameter space for a more sensitive fully coherent
follow up search.  
We tested this approach on 200 simulations in Gaussian noise, generating posterior distributions for the parameters described above. 
In 90\% of these simulations we limit the sky area to 45 deg$^2$ with a 95\% confidence contour.
However, find that this contour contains the true parameter only 42\% of the time.
We present these results and describe the features and shortcomings of our approach.


% Lines chapter
As mentioned above, we limit the effect of instrumental lines on the SOAP
search using machine learning, however we can also identify and mitigate these
lines separately before a search is run.  We demonstrate how we can use SOAP in
a simple configuration to identify instrumental lines.  We compare this method
to existing line identification tools used in the \gls{LIGO} collaboration, and
find that using the Viterbi statistic SOAP identifies $\sim 37$\% of the same lines as these methods, where for many of the lines which were not identified, other SOAP outputs do show evidence of a line.
With further investigation, we expect to identify many more lines in common with existing methods.
As well as these common lines, the SOAP algorithm returned $\sim 150$ more bands which potentially contain an instrumental line, which did not appear on \gls{LIGO} line-lists.








